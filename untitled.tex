
\begin{document}

\abstract


\ldots

\section{Introduction}

Taxonomy, the study of the classification of organisms has been developing and accumulating information for the past ... years. Contrary to what many people may think, it is a branch of knowledge in constant evolution, also in conjunction with scientific breakthrough, such as molecular phylogeny. 

It is customary to distinguish two levels in which scientific taxonomic consist:
\item the modelling of the hierarchical organization of the taxonomies, e.g., the organization of ranks, and 
\item the various lists of species identified by different authors/approach to taxonomy.

Taxonomic information is not only contained in dedicated species checklists, but is also present in a variety of resources used to describe data, be that statistics (e.g., reference data) or textual documents (e.g., thesauri).  

The traditional approach is to define one's own list of species, to be used in conjunction with the data collection privately managed by teh respective "owner" of the data. Problems from this? Explain..   

But with the constant increase of data on the web, and especially with the linked data approach to publishing data on the web, 

In this paper we focus on the management of taxonomic information in thesauri, paying special attention to the management of the ranking information, as well as the management of both scientific and common names. 

Many different works focus on either or both of these point  
ic information has been, aiming at developing conclusive taxonomies (ref), identifying the best ranks needed to describe taxonomies (ref), or also to identify the criteria that should lead classifications (rdf), to mention only a few direction of work and research.  

You can get started by \textbf{double clicking} this text block and begin editing. You can also click the \textbf{Insert} button below to add new block elements. Or you can \textbf{drag and drop an image} right onto this text. Happy writing!

\section{Related work}
\subsection{This is a Subsection}

In this paragraph I test citations \cite{Jentzsch_2014}.

\section{Proposal}

\section{Conclusions}

\bibliographystyle{unsrt}
\bibliography{my_bib_file}

\end{document}